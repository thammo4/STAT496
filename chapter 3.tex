\documentclass{article}
\usepackage{amsmath, amssymb, xpatch, amsthm, outlines}
\usepackage[utf8]{inputenc}

\theoremstyle{plain}

\makeatletter
\xpatchcmd{\@thm}{\thm@headpunct{.}}{}{}

\newtheorem*{def311*}{Definition 3.1.1}
\newtheorem*{def313*}{Definition 3.1.3}
\newtheorem*{def312*}{Definition 3.1.2}

\newtheorem*{lemma311*}{Lemma 3.1.1}

\newtheorem*{theorem311*}{Theorem 3.1.1}
\newtheorem*{theorem312*}{Theorem 3.1.2}
\newtheorem*{theorem313*}{Theorem 3.1.3}
\newtheorem*{theorem314*}{Theorem 3.1.4}

\newtheorem*{lemma312*}{Lemma 3.1.2}

\newtheorem*{definition321*}{Definition 3.2.1}
\newtheorem*{definition322*}{Definition 3.2.2}

\newtheorem*{theorem321*}{Theorem 3.2.1}
\newtheorem*{theorem322*}{Theorem 3.2.2}
\newtheorem*{theorem323*}{Theorem 3.2.3}

\newtheorem*{definition331*}{Definition 3.3.1}
\newtheorem*{definition332*}{Definition 3.3.2}
\newtheorem*{theorem331*}{Theorem 3.3.1}
\newtheorem*{definition333*}{Definition 3.3.3}
\newtheorem*{theorem332*}{Theorem 3.3.2}

\newtheorem*{theorem341*}{Theorem 3.4.1}
\newtheorem*{theorem342*}{Theorem 3.4.2}
\newtheorem*{lemma341*}{Lemma 3.4.1}
\newtheorem*{theorem343*}{Theorem 3.4.3}
\newtheorem*{lemma342*}{Lemma 3.4.2}
\newtheorem*{lemma343*}{Lemma 3.4.3}
\newcommand{\Lagr}{L}
\begin{document}
% \end{document}

$\newline$
\textbf{Definition: Metric Space} Space equipped with a norm $ || \cdot || $. Functions that are elements of the metric space must satisfy the following properties: \vspace{.35 em} \\
1. All functions have positive length save for the zero function. \vspace{.25 em} \\
2. The zero function has zero length. \vspace{.25 em} \\
3. Triangle Inequality: the length of the sum of two functions is always lesser than or equal to the sum of their individual lengths.

$\newline$
Equivalently:
\begin{align*}
& \text{\rm{1. }} \forall f : || f || \geq 0 \\
& \text{\rm{2. }} f = 0 \Leftrightarrow || f || = 0 \\
& \text{\rm{3. }} ||f - g || = || g - f || \\
& \text{\rm{4. }} || f + g || \leq || f || + || g ||
\end{align*}




\section*{The Spaces $L^{p}$}

\subsection*{Measurable Functions}




$\newline$
\textbf{Example 3.1.1 \hspace{.5 em} The Vitali Set $ V $}

\begin{def311*} \text{\rm{\textbf{Measurable Function}}} \\
Let $ f : \mathbb{R} \to \mathbb{R} $ and $ g : \mathbb{R} \to \mathbb{R^{+}} $. \\
Define a function $ \text{\rm{mid}}\left\{-g,f,g\right\} $ to be the unique range value chosen from $ \left\{-g,f,g\right\} $ that is between the other two functions in the triplet. $ f $ is a measurable function when:
$$ \forall g \in L^{1} \left[0, \infty\right) : \text{\rm{mid}}\left\{-g, f, g\right\} $$
\end{def311*}

%$\newline$
\begin{theorem311*} \textbf{Properties of Real, Measurable Functions} \\
Let $ f : \mathbb{R} \to \mathbb{R} $. Then:
\begin{align*}
& \text{\rm{1. }} f \text{\rm{ integrable }} \hspace{.52 em} \implies f \text{\rm{ measurable }} \\
& \text{\rm{2. }} f \text{\rm{ continuous }} \hspace{.18 em} \implies f \text{\rm{ measurable }} \\
& \text{\rm{3. }} f \text{\rm{ measruable }} \implies |f| \text{\rm{ measurable }} \\
& \text{\rm{4. }} f \text{\rm{ measurable }} \implies \forall p \in \mathbb{Z^{+}} : |f|^{p} \text{\rm{ measurable }}
\end{align*}
\end{theorem311*}

%$\newline$
\begin{theorem312*} Assume f is measurable, and let $ g $ be a Lebesgue integrable function. If $ |f| \leq g $, then $ f $ is Lebesgue integrable. Furthermore, if $ | f | $ is Lebesgue integrable, then $ f $ is too.
\end{theorem312*}

\subsection*{Definition of $L^{p}$ Spaces}

%$\newline$

%$\newline$
\begin{def312*} \text{\rm{\textbf{$L^{p}$ Function Space}}} \\
Let $ p \in \mathbb{R^{+}} : p \geq 1 $. The $ L^{p} $ function space consists of all measurable functions $ f $ having a finite $ L^{p} $ norm. Symbolically:
$$
L^{p} = \left\{ f: f \text{\rm{ measurable }} \hspace{.1 em} \land \hspace{.30 em}  || f ||_{p} < \infty \right\} ; \quad || f ||_{p} = \left(\int |f|^{p}\right)^{\frac{1}{p}}
$$
Two $ L^{p} $ functions equal a.e. represent the same function with respect to the $ L^{p} $ space.
\end{def312*}

$\newline$
\textbf{Definition: $L^{p}$ Distance Between Two Functions} \\
The $ L^{p} $ distance between two function space elements is:
$$
|| f - g||_{p} = \left( \int | f-g |^{p} \right)^{\frac{1}{p}}
$$

$ \newline $
\textbf{Definition: $L^{p} \left(a,b\right)$ Norm} \\
Let $ \left(a,b\right) \subseteq \mathbb{R} $. The $ L^{p}\left(a,b\right) $ norm is:
$$ ||f||_{p} = \left( \int\limits_{a}^{b} |f|^{p} \right)^{\frac{1}{p}} $$

%$\newline$
\begin{lemma311*} \text{\rm{\textbf{Hölder's Inequality}}} \\
Assume $ p, q > 1 $ satisfy $ \frac{1}{p} + \frac{1}{q} = 1 $. Then:
$$ \forall f, g : f \in L^{p} \hspace{.1 em} \land \hspace{.1 em} g \in L^{q} \hspace{.125 em} \implies \hspace{.125 em} f \cdot g \in L^{1} \hspace{.1 em} \land \hspace{.1 em} \int | f \cdot g | \leq || f ||_{p} \cdot || g ||_{q}
$$
\end{lemma311*}

\begin{theorem313*} \text{\rm{\textbf{To Define an}}} $L^{p}$ \text{\rm{\textbf{Norm}}} \\
On an $ L^{p} $ space with $ p \geq 1 $, we can define a norm with:
$$ || f ||_{p} = \left( \int | f |^{p}\right)^{\frac{1}{p}} $$
\end{theorem313*}

%\subsection*{Completeness of $ L^{p} $ Spaces}
\subsection*{The $L^{p}$ Spaces Are Complete}
\begin{def313*} \text{\rm{\textbf{Convergence of $ L^{p} $ Functions to Norm Limit}}} \\
A sequence of $ L^{p} $ functions $ \left\{f_{n}\right\}_{n=1}^{\infty} $ converges to a limit function $ f \in L^{p} $ when:
$$
\forall \epsilon \in \mathbb{R^{+}}, \exists N \in \mathbb{Z^{+}}, \forall n \in \mathbb{Z^{+}} : n > N \implies || f_{n} - f ||_{p}
$$
Then we say that $ f = \lim\limits_{n \to \infty} f_{n} $.
\end{def313*}

\begin{lemma312*}
Suppose $ \forall k \in \mathbb{Z^{+}} : f_{k} \in L^{p} $. If the real valued series $ \sum || f_{k} || $ converges, then:
\begin{itemize}
\item $ \exists g \in L^{p} : \sum f_{k} \to g $
\item $ \forall x \in \mathbb{R} : \left(\sum f_{k} \to g\right)_{\text{a.e.}} $ pointwise.
\end{itemize}
\end{lemma312*}




\begin{theorem314*} \text{\rm{\textbf{The Riesz-Fischer Theorem}}} \\
Let $ \left\{f_{n}\right\} $ be a Cuahcy sequence of functions in an $ L^{p}$ space with $ p \geq 1 $, i.e.
$$ \forall \epsilon \in \mathbb{R^{+}}, \exists N \in \mathbb{Z^{+}} : \forall m \in \mathbb{Z^{+}}, m \geq N \land n \geq N \implies ||f_{n} - f_{m}||_{p} $$
Then $ \exists f \in L^{p} : f_{n} \to f $ with respect to the $L^{p}$-norm limit.
\end{theorem314*}

 

\section*{$L^{2}$ and $\ell^{2} $ Hilbert Space Properties}

\begin{definition321*} \text{\rm{\textbf{Inner Product of}}} $L^{2}\left(\mathbb{R}\right)$ \text{\rm{\textbf{Functions}}} \\
The inner product of $ f \in L^{2}\left(\mathbb{R}\right) $ and $ g \in L^{2}\left(\mathbb{R}\right) $ is:
$$ \langle f,g \rangle = \int\limits_{-\infty}^{\infty} f\left(x\right) g\left(x\right) dx $$
\end{definition321*}

$\newline$
\textbf{Fundamental Properties of $ L^{2} $ Inner Product} \\
Let $ f, g, h \in L^{2} $. Let $ a, b \in \mathbb{R} $. Then:
\begin{align*}
& \text{\rm{1. }} \langle f, f \rangle = ||f||^{2} \\
& \text{\rm{2. }} \langle f, f \rangle \geq 0 \\
& \text{\rm{3. }} \langle f, g \rangle = \langle g, f \rangle \\
& \text{\rm{4. }} \langle af+bg, h \rangle = a\langle f, h \rangle + b \langle g, h \rangle \\
& \text{\rm{5. }} \langle f, f \rangle = 0 \Leftrightarrow f = 0 \\
\end{align*}


\begin{theorem321*} \text{\rm{\textbf{The Schwarz Inequality}}} \\
$$ \forall f, g \in L^{2} : | \langle f, g \rangle | \leq || f || \cdot || g || $$
\end{theorem321*}
\noindent The Schwarz Inequality implies that: $$ \cos \theta = \frac{ \langle f,g \rangle}{ ||f|| ||g|| } $$
\noindent $ f \in L^{2} $ and $ g \in L^{2} $ are perpendicular when: $$ \langle f, g \rangle = 0 $$

\begin{theorem322*} \text{\rm{\textbf{The Pythagorean Theorem}}} \\
$$ \forall f \in L^{2}, \forall g \in L^{2} : \langle f, g \rangle \implies || f ||^{2} + || g ||^{2} = || f + g ||^{2} $$
\end{theorem322*}

\begin{theorem323*} \text{\rm{\textbf{The Parallelogram Law}}} \\
$$ \forall f \in L^{2}, \forall g \in L^{2} : || f + g ||^{2} + || f - g ||^{2} = 2 || f ||^{2} + 2 || g ||^{2} $$

\end{theorem323*}

\begin{definition322*} \text{\rm{\textbf{Inner Product of}}} $L^{2}\left(a,b\right)$ \text{\rm{\textbf{functions}}} \\
$$ \forall f \in L^{2}\left(a,b\right), \forall g \in L^{2}\left(a,b\right) : \langle f, g \rangle = \int\limits_{a}^{b} f\left(x\right) g\left(x\right) dx $$
The angle between $ f $ and $ g $ satisfies:
$$ || f || \cdot || g || \cdot \cos \theta = \langle f, g \rangle $$ 
\end{definition322*}

\section*{Hilbert Space Orthonormal Basis}

\textbf{Definition: Hilbert Space} \\
A normed, linear, inner product space. \\
Possess properties of Euclidean space which allow for application of calculus.

\subsection*{Vector Space Basis}

\begin{theorem331*} \text{\rm{\textbf{Orthogonal Basis for Hilbert Space}}}
\end{theorem331*}

\subsection*{Gram-Schmidt Process}

%\begin{definition332*} \text{\rm{\textbf{The Gram-Schmidt Process}}}
%\end{definition322*}

\begin{definition332*} \text{\rm{\textbf{The Gram\--Schmidt Process}}}
\end{definition332*}


\subsection*{The Projection Theorem}

\begin{theorem331*} \text{\rm{\textbf{Basis Approximation Theorem}}} \\
\end{theorem331*}

\subsection*{Equivalence of Hilbert Spaces}

\begin{definition333*} \text{\rm{\textbf{Hilbert Space Isomorphism}}}
\end{definition333*}

\begin{theorem332*}
\end{theorem332*}

\section*{Application: Quantum Mechanics}
\subsection*{The Hermite Differential Equation}
\subsection*{The Hermite Functions Form an Orthogonal Basis for $ L^{2}\left(\mathbb{R}\right) $}



















































































\end{document}
