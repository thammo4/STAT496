\documentclass{article}
\usepackage{amsmath, amssymb, xpatch, amsthm, outlines}
\usepackage[utf8]{inputenc}

\theoremstyle{plain}

\makeatletter
\xpatchcmd{\@thm}{\thm@headpunct{.}}{}{}

\newtheorem*{two_one_one_defa*}{Definition 2.1.1 (Supremum)}

\newcommand{\Lagr}{L}


\newtheorem*{def411*}{Definition 4.1.1 Lebesgue Measurable Set}

\newtheorem*{theorem411*}{Theorem 4.1.1 Addition Rule for Lebesgue Sets}
\newtheorem*{theorem412*}{Theorem 4.1.2 Addition Rule for Disjoint Lebesgue Sets}
\newtheorem*{theorem413*}{Theorem 4.1.3}

\newtheorem*{def421*}{Definition 4.2.1 The Lebesgue Integral Apropos Borel Measure $\mu$}

\newtheorem*{def431*}{Definition 4.3.1 Generalized Hilbert Space}


\newtheorem*{def441*}{Definition 4.4.1 Experiment / Sample Space}
\newtheorem*{def442*}{Definition 4.4.2 Random Variable}

\newtheorem*{def443*}{Definition 4.4.3 Borel Distribution}




\begin{document}


\section*{Lebesgue Measure}
\begin{def411*} \end{def411*}

$\newline$
\textbf{Definition: Borel Algebra}



\subsection*{Properties of the Lebesgue Measure}
\begin{theorem411*} \end{theorem411*}

\begin{theorem412*} \end{theorem412*}


\subsection*{Consideration of Other Ways to Define Measure}
\textbf{General Measure: Four Desirable Properties}

\section*{Lebesgue Integrals with Respect to Other Measures}

\begin{def421*} \end{def421*}

\section*{The Hilbert Space $ L^{2}\left(\mu\right) $}

\section*{Application: Probability}

\begin{def441*} \end{def441*}

\begin{def442*} \end{def442*}

\begin{def443*} \end{def443*}






\end{document}
